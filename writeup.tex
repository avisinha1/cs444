\documentclass[onecolumn, draftclsnofoot,10pt, compsoc]{IEEEtran}
\usepackage{graphicx}
\usepackage{url}
\usepackage{setspace}

\usepackage{geometry}
\geometry{textheight=9.5in, textwidth=7in}


\def \DocType{		WriteUp
				}
			
\newcommand{\NameSigPair}[1]{\par
\makebox[2.75in][r]{#1} \hfil 	\makebox[3.25in]{\makebox[2.25in]{\hrulefill} \hfill		\makebox[.75in]{\hrulefill}}
\par\vspace{-12pt} \textit{\tiny\noindent
\makebox[2.75in]{} \hfil		\makebox[3.25in]{\makebox[2.25in][r]{Signature} \hfill	\makebox[.75in][r]{Date}}}}
% 3. If the document is not to be signed, uncomment the RENEWcommand below
\renewcommand{\NameSigPair}[1]{#1}

%group members: Aviral Sinha, Omar Elgebaly
%%%%%%%%%%%%%%%%%%%%%%%%%%%%%%%%%%%%%%%
\begin{document}

\pagenumbering{arabic}


\section{Design}
Our solution first checks that the list is not empty. If it is, it fetches the next I/O request from the queue. If the list is empty, add the item anywhere in the queue.  Otherwise, generate a list request to iterate. If the request sector position is greater than current position, add the item to the list directly before the current request position via insertion sort. 
	
\section{Questions}

\begin{enumerate}
\item \textbf{What do you think the main point of this assignment is?}
The main point of this assignment was to enhance our understanding of the CPU scheduler. It also helped familiarize is with the VM and how the pieces come together. 

\item \textbf{How did you personally approach the problem? Design decisions, algorithm, etc.}
We modified the noop-iosched.c file to make it a shortest seek time first implementation. The scheduler needed to be aware of the positioning at all times since the behaviour is largely determined by the sector position relative to the current position. 

\item \textbf{How did you ensure your solution was correct? Testing details, for instance.}
We ensured our solution was correct by writing into a testfile. We changed the scheduler using the make menuconfig command. To ensure we had changed the scheduler, we echo'd statements into that file and read out the contents to the terminal to see if the contents matched the expected output. We also implemented print statements into our sstf-iosched.c file to track what is happening to the list and which sector number is dispatched.  
\item \textbf{What did you learn?}
We learned about the VM and how the VM chooses which scheduler to use. We learned about the basic structure of a scheduler and how it merges and sorts. We made a new scheduler available to the VM as well and figured out how to compile the scheduler. We also learned various facts in the process such as the advantages of different scheduling algorithms.

\end{enumerate}

\section{Version Control}
\subsubsection{Version Control (Link:https://github.com/avisinha1/cs444/commits/master)}
\begin{center}
	\begin{tabular}{ | l | c | r|}
		\hline
		Who & Work & Date \\
		\hline
		Omar & noop-iosched.c uploaded & 10/30/2017\\
		\hline
		Omar & sstf-iosched.c uploaded. Almost complete but needs comments & 10/30/2017\\
		\hline
		Omar & Finished sstf-iosched.c. Added comments & 10/30/2017 \\
		\hline
		Omar & Added Kconfig.iosched & 10/30/2017 \\
		\hline
		Omar & Added patch & 10/30/2017 \\
		\hline
		Omar & added tex and makefile & 10/30/2017\\
		\hline
	\end{tabular}
\end{center}	
	
	


\section{Work Log}

	October 25th 2017: Began reading assignment details 
	
	October 26th 2017: Understood what the problem was asking and started a development of a solution based on noop
	
	October 27th 2017: began implementation of LOOK algorithms
	
	October 28th 2017: Started documentation of our work so it could be added to the write up
	
	October 28th 2017: Began figuring out how to run the test and properly implement patch
	
	October 29th 2017: Started putting content into the latex file so it would be complete by the due date
	
	October 29th 2017: continued work on the patch and so it be prepared for testing
	
	October 30th 2017: Began testing of patch and getting the results
	
	October 30th 2017: Added data from version control log into the LaTex as well as completed linux patch
	
	October 30th 2017: Finished LaTex write up and turned in on TEACH as a tarball
	
	


\end{document}