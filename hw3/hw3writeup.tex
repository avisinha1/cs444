\documentclass[onecolumn, draftclsnofoot,10pt, compsoc]{IEEEtran}
\usepackage{graphicx}
\usepackage{url}
\usepackage{setspace}

\usepackage{geometry}
\geometry{textheight=9.5in, textwidth=7in}


\def \DocType{		WriteUp
				}
			
\newcommand{\NameSigPair}[1]{\par
\makebox[2.75in][r]{#1} \hfil 	\makebox[3.25in]{\makebox[2.25in]{\hrulefill} \hfill		\makebox[.75in]{\hrulefill}}
\par\vspace{-12pt} \textit{\tiny\noindent
\makebox[2.75in]{} \hfil		\makebox[3.25in]{\makebox[2.25in][r]{Signature} \hfill	\makebox[.75in][r]{Date}}}}
% 3. If the document is not to be signed, uncomment the RENEWcommand below
\renewcommand{\NameSigPair}[1]{#1}

%group members: Aviral Sinha, Omar Elgebaly
%%%%%%%%%%%%%%%%%%%%%%%%%%%%%%%%%%%%%%%
\begin{document}

\pagenumbering{arabic}


\section{Design}
The block device code is based off the Simple Block Device implementation. This project will be implemented as Linux Kernel Module. It will take the device block size, the key, and the device major number as parameters. These parameters will only be modified once the module has been initialized.  The device utilizes an initialization method that allocates memory for the device and the AES encryption cipher. It also sets up a cryptographic key from the module parameter. The block queue will dispatch requests through the ebd\_request function and then dispatches read/write commands to the ebd\_transfer function. ebd\_transfer takes in two things: a buffer and a flag. If the flag is set to read, it reads the requested blocks off the device, decrypts them, and writes them to the buffer. If the flag is set to write, it takes and splits the bytes from the buffer and encrypts them before writing it to the device's memory.
\section{Questions}

\begin{enumerate}
\item \textbf{What do you think the main point of this assignment is?}
The main point of this assignment is to help with understanding how to implement a linux kernel module. It also involves going through some terribly documented C code. This forced us to read through the code very carefully and improved our C comprehension skills. This assignment also served to familiarize us with the Linux device driver model as well as a cryptic encryption API. 

\item \textbf{How did you personally approach the problem? Design decisions, algorithm, etc.}
My initial approach to this problem was to read through chapter 16 of the Linux Device Drivers book edition that the professor suggested. Through here and this blog post (http://blog.superpat.com/2010/05/04/a-simple-block-driver-for-linux-kernel-2-6-31/), I learned about block devices. I decided to use the Simple Block Driver because it seemed easier to use than the sbull one. After that, we needed to figure out how we were going to use the Crypto API to encrypt our data. 

\item \textbf{How did you ensure your solution was correct? Testing details, for instance.}
The solution was tested for correctness based on kernel print statements of the hexadecimal dumps in their original state and their encrypted state. We also used grep to search through devices memory to ensure that the string from the file could not be found. 

\item \textbf{What did you learn?}
I learned how to create a linux module, decrypt read flags, and encrypt write flags. I also learned the basics of the Linux device driver model. More generally, I also realized the importance of good documentation more than ever in this assignment as it was very frustrating combing through the c code. 

\end{enumerate}

\section{Version Control}
\subsubsection{Version Control (Link:https://github.com/avisinha1/cs444/commits/master)}
\begin{center}
	\begin{tabular}{ | l | c | r|}
		\hline
		Who & Work & Date \\
		\hline
		Omar & original simple block driver & 11/15/2017\\
		\hline
		Avi & uploaded encrypted block driver. needs comments. & 11/15/2017\\
		\hline
		Omar & uploaded encrypted block driver w/ some comments. & 11/15/2017 \\
		\hline
		Omar & improved readability & 11/15/2017 \\
		\hline
		Avi & added some more comments & 11/15/2017 \\
		\hline
	\end{tabular}
\end{center}	

\section{Work Log}

	November 9th: Began looking at assignment details and understanding what its asking 
	
	November 10th: Looked at classnotes, lecture recordings, and reccommended readings to get a better understanding before beginning the implementation
	
	November 12th: Began writing RAM device driver and implementing linux crypto API
	
	November 13th: Started documentation of our work so it could be added to the write up
	
	November 13th: Began understanding how to run our patch within the VM environment so when we complete the implementation we can properly test it. 
	
	November 14th: Started putting contents into LaTex file for proper documentation
	
	November 14th: Completed implementation and preparing it for testing and debugging
	
	November 15th: Began running tests on kernel and figuring out how to combat various problems that would be coming up
	
	November 15th: Added data from version control log into LaTex as well as completed linux patch file
	
	November 15th: Finished LaTex write up and turned in on TEACH as a tarball
	

\end{document}